
\chapter{Additional Features}
Various features of {\ViennaMath}, which are not necessarily standard features of a symbolic math library, are covered in this chapter.
Additional feature requests should be sent to
\begin{center}
\texttt{viennamath-support$@$lists.sourceforge.net} 
\end{center}

  \section{\LaTeX{} Output}
Since {\ViennaMath} encourages a high-level description and manipulation of the underlying mathematical problem formulation in source code, it is natural to
generate \LaTeX{} code from {\ViennaMath} expressions for debugging purposes. The generated code can be copy\&paste'd to LaTeX rendering webpages or used for
the automatic generation of program log files in the form of a \LaTeX{} document.

All conversion is carried out by a separate converter object of type \lstinline|rt_latex_translator<InterfaceType>| as defined in
\lstinline|viennamath/manipulation/latex.hpp|. A convenience shortcut \lstinline|latex_translator| is available for the default runtime expression interface.
Conversion is triggered by providing the expression to be converted to the functor:
\begin{lstlisting}
 latex_translator  to_latex;

 expr f = sqrt( x + y );
 to_latex( f );     //returns the string '\sqrt{x_{0}+x_{1}}'
\end{lstlisting}
By default, variables are printed as $x_0$, $x_1$, etc. This and other output routines can be customized by using the \lstinline|customize()| member function
of the converter. For example, to print 'x' and 'y' instead of 'x\_\{0\}' and 'x\_\{1\}', the code
\begin{lstlisting}
 to_latex.customize(x, "x");
 to_latex.customize(y, "y");
\end{lstlisting}
is sufficient. Similar customizations can be applied for the output of types and features described in the remainder of this chapter.

\NOTE{The \LaTeX{} generator works with runtime expression types only. Thus, compiletime expression types need to be converted to runtime expression types
first.}

  \section{Function Symbols}


  \section{Integration Symbols}


  \section{Differential Operators}
